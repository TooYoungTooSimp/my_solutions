\documentclass[]{cpp}
\title{省选基础算法}
\author{雷宇辰}
\begin{document}
	\setcounter{page}{0}
	\maketitle
	\newpage
	\tableofcontents
	\newpage
	\setcounter{page}{1}
	\section{day1 图论}
	\subsection{有向图强连通分量的 Tarjan 算法}
	\paragraph{定义}
	在\textbf{有向图$G$}中,如果两个顶点$u,v$间存在一条路径$u$到$v$的路径且也存在一条$v$到$u$的路径,则称这两个顶点$u,v$是\textbf{强连通的(strongly connected)}。如果有向图$G$的每两个顶点都强连通,称$G$是一个\textbf{强连通图}。有向非强连通图的 极大强连通子图,称为\textbf{强连通分量(strongly connected components)}。若将有向图中的强连通分量都缩为一个点,则原图会形成一个 DAG(有向无环图)。
	\subparagraph{极大强连通子图}
	G 是一个极大强连通子图当且仅当 G 是一个强连通子图且不存在另一个强连通子图 G’使得 G 是 G’的真子集。
	\paragraph{Tarjan 算法}
	定义$dfn(u)$为结点$u$搜索的次序编号,给出函数$low(u)$使得\\
	$low(u) = min$
	\\$\{$\\
	\verb|    |$dfn(u),$\\
	\verb|    |$low(v),$  \quad $(u,v)$为树枝边,$u$为$v$的父结点\\
	\verb|    |$dfn(v)\;$ \quad $(u,v)$为后向边或指向栈中结点的横叉边
	\\$\}$\\
	当结点$u$的搜索过程结束后,若$dfn(u)=low(u)$,则以$u$为根的搜索子树上所有还在栈中的结点是一个强连通分量。
	\subparagraph{代码}$\\$
	\codeinput[tarjan - SCC]{assets/day1/tarjan-scc.cpp}
	\paragraph{练习题}
	\subparagraph{
	\href{http://poj.org/problem?id=2186}{POJ2186}/\href{http://www.lydsy.com/JudgeOnline/problem.php?id=1051}{BZOJ1051} - Popular Cows} 双倍的快乐
	\codeinput[Popular Cows]{assets/day1/poj2186.cpp}
	\subparagraph{\href{http://poj.org/problem?id=3180}{POJ3180} - The Cow Prom}
	\verb|The N (2 <= N <= 10,000) cows are so excited.|
	\codeinput[The Cow Prom]{assets/day1/poj3180.cpp}
	\subparagraph{\href{http://poj.org/problem?id=3180}{POJ1236} - Network of Schools}
	强连通分量缩点求出度为0的和入度为0的分量个数
	\codeinput[Network of Schools]{assets/day1/poj1236.cpp}
	\subsection{图的割点、桥与双连通分量}
	\paragraph{定义}
	\subparagraph{点连通度与边连通度}
	在一个\textbf{无向连通图}中,如果有一个顶点集合$V$,删除顶点集合$V$,以及与$V$中顶点相连(至少有一端在$V$中)的所有边后,原图\textbf{不连通},就称这个点集$V$为\textbf{割点集合}。\\
	一个图的\textbf{点连通度}的定义为:最小割点集合中的顶点数。\\
	类似的,如果有一个边集合,删除这个边集合以后,原图不连通,就称这个点集为\textbf{割边集合}。
	\subparagraph{双连通图、割点与桥}
	如果一个无向连通图的\textbf{点连通度大于$1$},则称该图是\textbf{点双连通的(point biconnected)},简称双连通或重连通。一个图有\textbf{割点},当且仅当这个图的点连通度为$1$,则割点集合的唯一元素被称为\textbf{割点(cut point)},又叫关节点(articulation point)。一个图可能有多个割点。\\
	如果一个无向连通图的\textbf{边连通度大于$1$},则称该图是\textbf{边双连通的(edge biconnected)},简称双连通或重连通。一个图有\textbf{桥},当且仅当这个图的边连通度为$1$,则割边集合的唯一元素被称为\textbf{桥(bridge)},又叫关节边(articulation edge)。一个图可能有多个桥。\\
	可以看出,点双连通与边双连通都可以简称为双连通,它们之间是有着某种联系的,下文中提到的双连通,均既可指点双连通,又可指边双连通。(但这并不意味着它们等价)\\
	双连通分量(分支):在图 G 的所有子图 G'中,如果 G'是双连通的,则称 G'为双连通子图。如果一个双连通子图 G'它不是任何一个双连通子图的真子集,则 G'为极大双连通子图。双连通分量(biconnected component),或重连通分量,就是图的极大双连通子图。特殊的,点双连通分量又叫做块。
	\paragraph{Tarjan 算法}
	给出函数$low(u)$使得\\
	$low(u) = min$
	\\$\{$\\
	\verb|    |$dfn(u),$\\
	\verb|    |$low(v),$  \quad $(u,v)$为树枝边(父子边)\\
	\verb|    |$dfn(v)\;$ \quad $(u,v)$为后向边(返祖边) 等价于$dfn(v)<dfn(u)$且$v$不为$u$的父亲结点
	\\$\}$\\
	\subparagraph{代码}$\\$
	\codeinput[tarjan - BCC]{assets/day1/tarjan-bcc.cpp}
	\paragraph{练习题}
	\subparagraph{\href{http://poj.org/problem?id=3177}{POJ3177} - Redundant Paths}
	将一张有桥图通过加边变成边双连通图,至少要加$\frac{leaf+1}{2}$条边。
	\codeinput[Redundant Paths]{assets/day1/poj3177.cpp}
	\subparagraph{\href{http://poj.org/problem?id=1523}{POJ1523} - SPF}
	求割点与删除这个点之后有多少个连通分量
	\codeinput[Redundant Paths]{assets/day1/poj1523.cpp}
\end{document}