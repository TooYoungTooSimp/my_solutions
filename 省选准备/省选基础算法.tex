\documentclass[]{cpp}
\title{省选基础算法}
\author{雷宇辰}
\begin{document}
\setcounter{page}{0}
\maketitle
\newpage
\tableofcontents
\newpage
\setcounter{page}{1}
\section{day1 图论}
\subsection{有向图强连通分量的 Tarjan 算法}
\paragraph{定义}
	在\textbf{有向图$G$}中,如果两个顶点$u,v$间存在一条路径$u$到$v$的路径且也存在一条$v$到$u$的路径,则称这两个顶点$u,v$是\textbf{强连通的(strongly connected)}。如果有向图$G$的每两个顶点都强连通,称$G$是一个\textbf{强连通图}。有向非强连通图的 极大强连通子图,称为\textbf{强连通分量(strongly connected components)}。若将有向图中的强连通分量都缩为一个点,则原图会形成一个 DAG(有向无环图)。
\subparagraph{极大强连通子图}
	$G$是一个极大强连通子图当且仅当$G$是一个强连通子图且不存在另一个强连通子图$G'$使得$G$是$G'$的真子集。
\paragraph{Tarjan 算法}
	定义$dfn(u)$为结点$u$搜索的次序编号,给出函数$low(u)$使得\\
	$low(u) = min$
	\\$\{$\\
	\verb|    |$dfn(u),$\\
	\verb|    |$low(v),$  \quad $(u,v)$为树枝边,$u$为$v$的父结点\\
	\verb|    |$dfn(v)\;$ \quad $(u,v)$为后向边或指向栈中结点的横叉边
	\\$\}$\\
	当结点$u$的搜索过程结束后,若$dfn(u)=low(u)$,则以$u$为根的搜索子树上所有还在栈中的结点是一个强连通分量。
\subparagraph{代码}$\\$
\codeinput[tarjan - SCC]{assets/day1/tarjan-scc.cpp}
\paragraph{练习题}
\subparagraph{\href{http://poj.org/problem?id=2186}{POJ2186}/\href{http://www.lydsy.com/JudgeOnline/problem.php?id=1051}{BZOJ1051} - Popular Cows} 双倍的快乐
\codeinput[Popular Cows]{assets/day1/poj2186.cpp}
\subparagraph{\href{http://poj.org/problem?id=3180}{POJ3180} - The Cow Prom} The $N (2 <= N <= 10,000)$ cows are so excited.
\codeinput[The Cow Prom]{assets/day1/poj3180.cpp}
\subparagraph{\href{http://poj.org/problem?id=3180}{POJ1236} - Network of Schools} 强连通分量缩点求出度为0的和入度为0的分量个数
\codeinput[Network of Schools]{assets/day1/poj1236.cpp}
\subsection{图的割点、桥与双连通分量}
\paragraph{定义}
\subparagraph{点连通度与边连通度}
	在一个\textbf{无向连通图}中,如果有一个顶点集合$V$,删除顶点集合$V$,以及与$V$中顶点相连(至少有一端在$V$中)的所有边后,原图\textbf{不连通},就称这个点集$V$为\textbf{割点集合}。\\
	一个图的\textbf{点连通度}的定义为:最小割点集合中的顶点数。\\
	类似的,如果有一个边集合,删除这个边集合以后,原图不连通,就称这个点集为\textbf{割边集合}。
\subparagraph{双连通图、割点与桥}
	如果一个无向连通图的\textbf{点连通度大于$1$},则称该图是\textbf{点双连通的(point biconnected)},简称双连通或重连通。一个图有\textbf{割点},当且仅当这个图的点连通度为$1$,则割点集合的唯一元素被称为\textbf{割点(cut point)},又叫关节点(articulation point)。一个图可能有多个割点。\\
	如果一个无向连通图的\textbf{边连通度大于$1$},则称该图是\textbf{边双连通的(edge biconnected)},简称双连通或重连通。一个图有\textbf{桥},当且仅当这个图的边连通度为$1$,则割边集合的唯一元素被称为\textbf{桥(bridge)},又叫关节边(articulation edge)。一个图可能有多个桥。\\
	可以看出,点双连通与边双连通都可以简称为双连通,它们之间是有着某种联系的,下文中提到的双连通,均既可指点双连通,又可指边双连通。(但这并不意味着它们等价)\\
	双连通分量(分支):在图$G$的所有子图$G'$中,如果$G'$是双连通的,则称$G'$为双连通子图。如果一个双连通子图$G'$它不是任何一个双连通子图的真子集,则$G'$为极大双连通子图。双连通分量(biconnected component),或重连通分量,就是图的极大双连通子图。特殊的,点双连通分量又叫做块。
\paragraph{Tarjan 算法}
	给出函数$low(u)$使得\\
	$low(u) = min$
	\\$\{$\\
	\verb|    |$dfn(u),$\\
	\verb|    |$low(v),$  \quad $(u,v)$为树枝边(父子边)\\
	\verb|    |$dfn(v)\;$ \quad $(u,v)$为后向边(返祖边) 等价于$dfn(v)<dfn(u)$且$v$不为$u$的父亲结点
	\\$\}$\\
\subparagraph{代码}$\\$
\codeinput[tarjan - BCC]{assets/day1/tarjan-bcc.cpp}
\paragraph{练习题}
\subparagraph{\href{http://poj.org/problem?id=3177}{POJ3177} - Redundant Paths}
将一张有桥图通过加边变成边双连通图,至少要加$\frac{leaf+1}{2}$条边。
\codeinput[Redundant Paths]{assets/day1/poj3177.cpp}
\subparagraph{\href{http://poj.org/problem?id=1523}{POJ1523} - SPF}
求割点与删除这个点之后有多少个连通分量
\codeinput[Redundant Paths]{assets/day1/poj1523.cpp}
\subparagraph{\href{http://poj.org/problem?id=2942}{POJ2942} - Knights of the Round Table}
这题过于复杂,我来先给个\href{http://blog.csdn.net/lyy289065406/article/details/6756821}{别人的题解}。然后是我自己的实现(仿佛还是没看懂。
\\实现被狗吃了
%\codeinput[Knights of the Round Table]{assets/day1/poj2942.cpp}
\subsection{2-SAT}
\paragraph{定义}
	给定一个布尔方程,判断是否存在一组布尔变量的取值方案,使得整个方程值为真的问题,被称为布尔方程的可满足性问题(SAT)。SAT问题是NP完全的,但对于一些特殊形式的SAT问题我们可以有效求解。\\
	我们将下面这种布尔方程称为合取范式:
	$$(a\lor b\lor c\lor\cdots)\land(d\lor e\lor f\lor\cdots)\land\cdots$$
	其中$a,b,c,\cdots$称为文字,它是一个布尔变量或其否定。像$(a\lor b\lor c\lor\cdots)$这样用$\lor$连接的部分称为子句。如果合取范式的每个子句中的文字个数都不超过两个,那么对应的SAT问题又称为\textbf{2-SAT}问题。
\paragraph{解法}
	对于给定的\textbf{2-SAT}问题,首先利用$\Rightarrow$将每个子句$(a\lor b)$改写成等价形式$(\neg a\Rightarrow b\land a\Rightarrow\neg b)$.这样原布尔公式就变成了把$a\Rightarrow b$形式的布尔公式用$\land$连接起来的形式。\\
	对每个布尔变量$x$构造两个顶点分别代表$x$与$\neg x$。以$\Rightarrow$关系为边建立有向图。若在此图中$a$点能到达$b$点,就表示$a$为真时$b$也一定为真。因此该图中同一个强连通分量中所含的所有变量的布尔值均相同。\\
	若存在某个变量$x$,代表$x$与$\neg x$的两个顶点在同一个强连通分量中,则原布尔表达式的值无法为真。\\
	反之若不存在这样的变量,那么我们先将原图中所有的强连通分量缩为一个点,构出一个新图,新图显然是一个拓扑图,我们求出它的一个拓扑序。那么对于每个变量$x$,\textbf{$$x\text{所在的强连通分量(新图中的点)的拓扑序在}\neg x\text{所在的强连通分量之后}\Leftrightarrow x\text{为真}$$}就是一组合适布尔变量赋值。注意到 Tarjan 算法所求的强连通分量就是按拓扑序的逆序得出的,因此不需要真的缩点建新图求拓扑序,直接利用强连通分量的编号来当做顺序即可。
\paragraph{练习题}
\subparagraph{\href{http://poj.org/problem?id=3648}{POJ3648} - Wedding}
Additionally, there are several pairs of people conducting adulterous relationships (both different-sex and same-sex relationships are possible)
\codeinput[adulterous relationships]{assets/day1/poj3648.cpp}
\subparagraph{\href{http://poj.org/problem?id=3678}{POJ3678} - Katu Puzzle} 我什么时候做过这个题?
\codeinput[Katu Puzzle]{assets/day1/poj3678.cpp}
\subparagraph{\href{http://poj.org/problem?id=2749}{POJ2749} - Building roads} 杀光奶牛问题就会得到解决
\codeinput[Building roads]{assets/day1/poj2749.cpp}
\subsection{欧拉回路}
\paragraph{定义}
	设$G=(V,E)$是一个图。
\subparagraph{欧拉回路} 图$G$中经过\textbf{每条边一次}并且\textbf{仅一次}的回路称作欧拉回路。
\subparagraph{欧拉路径} 图$G$中经过\textbf{每条边一次}并且\textbf{仅一次}的路径称作欧拉路径。
\subparagraph{欧拉图}   存在\textbf{欧拉回路}的图称为欧拉图。
\subparagraph{半欧拉图} 存在欧拉路径但不存在欧拉回路的图称为半欧拉图。
\paragraph{性质与定理}
	以下不加证明的给出一些定理\quad\sout{(因为我懒得抄讲义了}
\subparagraph{定理 1} 无向图$G$为欧拉图,当且仅当$G$为连通图且所有顶点的度为偶数。
\subparagraph{推论 1} 无向图$G$为半欧拉图,当且仅当$G$为连通图且除了两个顶点的度为奇数之外,其它所有顶点的度为偶数。
\subparagraph{定理 2} 有向图$G$为欧拉图,当且仅当$G$的基图\footnote{忽略有向图所有边的方向,得到的无向图称为该有向图的基图。}连通,且所有顶点的入度等于出度。
\subparagraph{推论 2} 有向图$G$为半欧拉图,当且仅当$G$的基图连通,且存在顶点$u$的入度比出度大1、$v$的入度比出度小1,其它所有顶点的入度等于出度。
\paragraph{解法}
	由此可以得到以下求欧拉图$G$的欧拉回路的算法:
	\begin{enumerate}
		\item 在图$G$中任意找一个回路$C$。
		\item 将图$G$中属于回路$C$的边删除
		\item 在残留图的各极大连通子图中分别寻找欧拉回路。
		\item 将各极大连通子图的欧拉回路合并到$C$中得到图$G$的欧拉回路。
	\end{enumerate}
	该算法的伪代码如下:
	\begin{verbatim}
	void dfs(u)
	{
	    for (edge e : edges[u])
	        if (!flag[e])
	        {
	            flag[e] = true;
	            flag[rev(e)] = true; //如果图 G 是有向图则删去本行
	            dfs(e.to);
	            S.push(v);
	        }
	}
	\end{verbatim}
	最后依次取出栈$S$每一条边而得到图$G$的欧拉回路(也就是边出栈序的逆序)。由于该算法执行过程中每条边最多访问两次,因此该算法的时间复杂度为$O(|E|)$。
\paragraph{练习题}
\subparagraph{\href{http://uoj.ac/problem/117}{UOJ117} - 欧拉回路}
	混合两个子任务使代码风格变得鬼畜起来。
\codeinput[Building roads]{assets/day1/uoj117.cpp}
\section{day2 字符串(一)}
\subsection{KMP}
\paragraph{算法介绍} 用来在线性时间内匹配字符串
\paragraph{算法流程}
	我觉得鏼鏼鏼在WC上讲的比较清楚,于是我开始抄讲义。\\
	\textbf{字符串:} $s[1\ldots n]$, $|s| = n$。\\
	\textbf{子串:} $s[i\ldots j] = s[i]s[i + 1]\cdots[j]$。\\
	\textbf{前缀:} $pre(s,x) = s[1\ldots x]$, 后缀:$suf(s,x) = s[n − x + 1\ldots n]$\\
	若 $0 \leq r \le |s|, pre(s,r) = suf(s,r)$, 就称 $pre(s,r)$ 是 $s$ 的 \textbf{border}。\\
	KMP算法的第一步主要做这么一件事:在$O(n)$时间求出数组$next[1\ldots n]$, 其中$next[i]$表示前缀$s[1\ldots i]$的最大 border 长度。
	于是可以知道$s$的所有 border 长度为${next[n],next[next[n]],\cdots}$,我想这是显然的,于是不加证明的在这里给出。\\
	第二步就是匹配,如果失配了就把模式串的当前位置指针$i$跳到$next[i]$处然后继续匹配,然后就好了。
\paragraph{算法实现}$\\$
\codeinput[genNext]{assets/day2/kmp-genNext.cpp}
\codeinput[Find]{assets/day2/kmp-Find.cpp}
\paragraph{练习题}
\subparagraph{\href{http://poj.org/problem?id=3461}{POJ3461} - Oulipo} 求出所有匹配位置
\codeinput[Oulipo]{assets/day2/poj3461.cpp}
\subparagraph{\href{http://poj.org/problem?id=2406}{POJ2406} - Power Strings} $next$数组的奇妙性质
\codeinput[Power Strings]{assets/day2/poj2406.cpp}
\subparagraph{\href{http://codeforces.com/problemset/problem/526/D}{CF526D} - Om Nom and Necklace} 啥?
\codeinput[Om Nom and Necklace]{assets/day2/CF526D.cpp}
讲道理我KMP真的学的不是很明白,望各位dalao给予指导。
\subsection{Trie}
\paragraph{简介}
	字典树,也称 Trie、字母树,指的是某个字符串集合对应的形如下图的有根树。树的每条边上对应有恰好一个字符,每个顶点代表从根到该节点的路径所对应的字符串(将所有经过的边上的字符按顺序连接起来)。
\paragraph{实现} 水
\codeinput[Trie - impl]{assets/day2/trieImpl.cpp}
\paragraph{练习题}
\subparagraph{\href{http://poj.org/problem?id=3630}{POJ3630} - Phone List}
若插入过程中,有某个经过的节点带有串结尾标记,则之前插入的某个串是当前串的前缀。
\codeinput[Oulipo]{assets/day2/poj3630.cpp}
\subparagraph{\href{http://poj.org/problem?id=2945}{POJ2945} - Find the Clones} 
$n$个基因片段,每个长度为$m$,输出$n$行表示重复出现$i$次$(1 \leq i \leq n)$的基因片段的个数
\codeinput[Find the Clones]{assets/day2/poj2945.cpp}
待续
\subsection{Aho–Corasick Automaton}
\paragraph{简介}
	多模式串字符串匹配,Trie上的KMP,其中$next$数组变成了$fail$指针,功能相同。
\paragraph{实现}
	Trie的实现上文已经出现,所以此处不再重复。
\codeinput[buildFail]{assets/day2/AC-buildFail.cpp}
\paragraph{练习题}
\subparagraph{\href{http://acm.hdu.edu.cn/showproblem.php?pid=2222}{HDU2222} - Keywords Search}
AC 自动机模板题,注意统计答案时,每个节点只能统计一次不要重复统计。
\codeinput[Keywords Search]{assets/day2/hdu2222.cpp}
\subsection{Manacher}
求出字符串每一位的回文半径,算法流程奥妙重重,不易让常人理解,然后我就抄了份板子改了改,然后就比讲义上的标程快了$20\%$。
\paragraph{练习题}
\subparagraph{\href{http://poj.org/problem?id=3974}{POJ3974} - Palindrome}
Manacher 模板题
\codeinput[Palindrome]{assets/day2/poj3974.cpp}
\end{document}